\section{Dimensionality Reduction and Feature Selection}\label{sec:dimred}

Feature selection techniques are methods used to eliminate features which are redundant or irrelevant. This is achieved using different methods but the most common is to find any features which are correlated to each other (or correlation with the outcome) thus removing redundant and unwanted features. This in turn can simplify the hypothesis in a model which can improve both performance and memory allocation, while also reduces overfitting (reducing bias).  There are various techniques to achieve this but for the purpose of this project, Recursive Feature Elimination (RFE) and Chi2 feature selection will be described.

\noindent On the other hand, dimensionality reduction, reduces the dimensionality by creating new synthetic features based on the original features using some method unlike feature selection which chooses a subset of the original features. This is done by reducing multiple correlated features/dimensions into fewer synthetic features/dimensions, like compressing 2 dimensions into 1 dimension.  This has the same ramifications as feature selection (performance and memory) but except for trying to keep as much information as possible.  It is also used to help visualize multi-dimensional datasets in plots.   
 
\subsection{Recursive Feature Elimination}\label{ssec:rfe}
The Recursive Feature Elimination \cite{guyon2002gene}, as the name suggests, recursively fits the model and eliminates the least important feature/s with every iteration. This is done by ranking the features according to their coefficient weight and eliminating the least weighted feature/s, depending if the algorithm is set to eliminate 1 or more feature with every iteration. After each iteration the model is fitted again, and the least weighted feature/s are eliminated again, until the specified number of maximum iterations or features to be eliminated is reached. In this project, this will be used with the Logistic Regression classifier.                                                                    

\subsection{Chi2 Feature Selection}\label{ssec:chisquared}
The Chi2 test \cite{jin2006machine} is used to determine if the feature/s and outcome are dependent or not. It is important that the features and classes do not contain non-negative values (re-scaling can be used to change negative values to positive values, by changing the range of the values as described in Subsection \ref{ssec:normalisation}). It is often used in Machine Learning to rank features based on their Chi2 statistic (score). Features which are not found to be dependent on the outcome (irrelevant for classification), are discarded depending on a specified number of features to be selected, so that only the top ranked features are selected. 
\subsection{Principal Component Analysis}\label{ssec:pca}
Principal Component Analysis \cite{tipping1999probabilistic}, or PCA for short, is a dimensionality reduction technique where its main objective, as the name suggests, is to identify principal components in a dataset. Principal components are identified with the help of a covariance matrix as shown in Equation \eqref{eq:pcacovariance} of the dataset. When two or more variables have a high positive covariance, the information in both can be represented by a single Principal Component.

\begin{multicols}{2}
    \begin{equation} \label{eq:pcacovariance}
        \Sigma = \sum\limits_{i=1}^{n} (x_i) (x_i) ^T
    \end{equation} 
    \\ 
    \begin{equation} \label{eq:pcasvd}
       \Sigma = U \cdot S \cdot V^T
    \end{equation}
\end{multicols}

\noindent After calculating the covariance matrix, the eigenvectors and their eigenvalues are calculated by decomposing the covariance matrix. This is done by several methods, but the most commonly used method is Singular-Value Decomposition, or SVD for short. What SVD does as shown in Equation \eqref{eq:pcasvd}, is decompose $m \ \times \ n$ into 3 matrices, $m \ \times \ m$ unitary matrix $U$, an $m \ \times \ n$ diagonal matrix $S$ and an $n \ \times \ n$ unitary matrix $V$. From this decomposition, the eigevalues and eigenvectors are extracted. This process have been simplified for the purposes of this project. 

\noindent The eigenvectors represent the direction of the new axis’ and the eigenvalues represent the variance of data on that eigenvector. The eigenvectors are then sorted by their eigenvalues, and the eigenvectors that have the least variance can be discarded. For a k-dimensional dataset, there can only be k-eigenvectors, as all the combination of eigenvectors must span over the k-dimensional space. Therefore, when eliminating the eigenvectors that represent the data with least variance, dimensions are being reduced, as their data can still be projected with minimal error on the eigenvectors that represent the data with most variance.

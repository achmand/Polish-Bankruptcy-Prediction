\subsection{Random Forest}\label{ssec:implrf}
The code for Random Forest can be found in \textbf{random\_forest.py}. This code imports \textbf{decision\_tree.py} since it uses a collection of \textit{DecisionTree} to build to model. This script has a class named \textit{RandomForest} and it takes the following arguments in the constructor; \textit{n\_estimators} (number of Decision Trees), \textit{max\_features} (number of features to be selected randomly), \textit{bag\_ratio} (the percentage of the total number of instances to be used to bag the instances), \textit{bag\_features} (boolean to check if bagging on the dataset will be used), and \textit{random\_seed} (to seed the random instance). 

\noindent When an instance of \textit{RandomForest} is initialized the \textit{fit} function can be called to start training the model. This function takes the instances $X$ as a parameter and their labels $y$. The \textit{fit} function then calls the \textit{construct\_forest} function, and iterates for the number of \textit{n\_estimators} passed. In each iteration the random subspace method is done based on the parameter passed as \textit{max\_features} and if \textit{bag\_features} is set to true, the dataset bagging is done based on the percentage passed in \textit{bag\_ratio}, then an instance of \textit{DecisionTree} is initialized and it's \textit{fit} function is called. All the initialized estimators are saved in a list. Once the training phase is completed the \textit{predict} function can be called which takes a set of instances $x$ as an argument and returns an array with the predicted outcomes (using the majority vote method since its a classification problem). 

\noindent For our experiments Random Forest was initialized using 10 Decision Trees with the same implementation described in Subsection  \ref{ssec:impldt},  with random subspace method (number of features selected using square root method) and dataset bagging (60\% of the total instances are selected randomly and passed to each estimator) all of which were described in Subsection \ref{ssec:randomforest}.
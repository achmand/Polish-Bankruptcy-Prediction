\subsubsection{Synthetic Minority Over-sampling Technique}\label{sssec:smote}

\noindent There are various algorithms one can use to generate synthetic data. For this project, SMOTE algorithm will be used to over-sample the dataset as it is one of the most well-known technique for oversampling. SMOTE or Synthetic Minority Over-sampling Technique \cite{chawla2002smote} works by selecting/sampling similar instances of the minority class, by finding the nearest neighbours \(k\) (using some distance metric based on the feature set) and changing each attribute one at time by multiplying each \(x\) by a random number (between 0 to 1), therefore creating a new synthetic instance.       